\PassOptionsToPackage{unicode=true}{hyperref} % options for packages loaded elsewhere
\PassOptionsToPackage{hyphens}{url}
%
\documentclass[]{article}
\usepackage{lmodern}
\usepackage{amssymb,amsmath}
\usepackage{ifxetex,ifluatex}
\usepackage{fixltx2e} % provides \textsubscript
\ifnum 0\ifxetex 1\fi\ifluatex 1\fi=0 % if pdftex
  \usepackage[T1]{fontenc}
  \usepackage[utf8]{inputenc}
  \usepackage{textcomp} % provides euro and other symbols
\else % if luatex or xelatex
  \usepackage{unicode-math}
  \defaultfontfeatures{Ligatures=TeX,Scale=MatchLowercase}
\fi
% use upquote if available, for straight quotes in verbatim environments
\IfFileExists{upquote.sty}{\usepackage{upquote}}{}
% use microtype if available
\IfFileExists{microtype.sty}{%
\usepackage[]{microtype}
\UseMicrotypeSet[protrusion]{basicmath} % disable protrusion for tt fonts
}{}
\IfFileExists{parskip.sty}{%
\usepackage{parskip}
}{% else
\setlength{\parindent}{0pt}
\setlength{\parskip}{6pt plus 2pt minus 1pt}
}
\usepackage{hyperref}
\hypersetup{
            pdftitle={Class 6: R functions},
            pdfauthor={Jasmin Revanna},
            pdfborder={0 0 0},
            breaklinks=true}
\urlstyle{same}  % don't use monospace font for urls
\usepackage[margin=1in]{geometry}
\usepackage{color}
\usepackage{fancyvrb}
\newcommand{\VerbBar}{|}
\newcommand{\VERB}{\Verb[commandchars=\\\{\}]}
\DefineVerbatimEnvironment{Highlighting}{Verbatim}{commandchars=\\\{\}}
% Add ',fontsize=\small' for more characters per line
\usepackage{framed}
\definecolor{shadecolor}{RGB}{248,248,248}
\newenvironment{Shaded}{\begin{snugshade}}{\end{snugshade}}
\newcommand{\AlertTok}[1]{\textcolor[rgb]{0.94,0.16,0.16}{#1}}
\newcommand{\AnnotationTok}[1]{\textcolor[rgb]{0.56,0.35,0.01}{\textbf{\textit{#1}}}}
\newcommand{\AttributeTok}[1]{\textcolor[rgb]{0.77,0.63,0.00}{#1}}
\newcommand{\BaseNTok}[1]{\textcolor[rgb]{0.00,0.00,0.81}{#1}}
\newcommand{\BuiltInTok}[1]{#1}
\newcommand{\CharTok}[1]{\textcolor[rgb]{0.31,0.60,0.02}{#1}}
\newcommand{\CommentTok}[1]{\textcolor[rgb]{0.56,0.35,0.01}{\textit{#1}}}
\newcommand{\CommentVarTok}[1]{\textcolor[rgb]{0.56,0.35,0.01}{\textbf{\textit{#1}}}}
\newcommand{\ConstantTok}[1]{\textcolor[rgb]{0.00,0.00,0.00}{#1}}
\newcommand{\ControlFlowTok}[1]{\textcolor[rgb]{0.13,0.29,0.53}{\textbf{#1}}}
\newcommand{\DataTypeTok}[1]{\textcolor[rgb]{0.13,0.29,0.53}{#1}}
\newcommand{\DecValTok}[1]{\textcolor[rgb]{0.00,0.00,0.81}{#1}}
\newcommand{\DocumentationTok}[1]{\textcolor[rgb]{0.56,0.35,0.01}{\textbf{\textit{#1}}}}
\newcommand{\ErrorTok}[1]{\textcolor[rgb]{0.64,0.00,0.00}{\textbf{#1}}}
\newcommand{\ExtensionTok}[1]{#1}
\newcommand{\FloatTok}[1]{\textcolor[rgb]{0.00,0.00,0.81}{#1}}
\newcommand{\FunctionTok}[1]{\textcolor[rgb]{0.00,0.00,0.00}{#1}}
\newcommand{\ImportTok}[1]{#1}
\newcommand{\InformationTok}[1]{\textcolor[rgb]{0.56,0.35,0.01}{\textbf{\textit{#1}}}}
\newcommand{\KeywordTok}[1]{\textcolor[rgb]{0.13,0.29,0.53}{\textbf{#1}}}
\newcommand{\NormalTok}[1]{#1}
\newcommand{\OperatorTok}[1]{\textcolor[rgb]{0.81,0.36,0.00}{\textbf{#1}}}
\newcommand{\OtherTok}[1]{\textcolor[rgb]{0.56,0.35,0.01}{#1}}
\newcommand{\PreprocessorTok}[1]{\textcolor[rgb]{0.56,0.35,0.01}{\textit{#1}}}
\newcommand{\RegionMarkerTok}[1]{#1}
\newcommand{\SpecialCharTok}[1]{\textcolor[rgb]{0.00,0.00,0.00}{#1}}
\newcommand{\SpecialStringTok}[1]{\textcolor[rgb]{0.31,0.60,0.02}{#1}}
\newcommand{\StringTok}[1]{\textcolor[rgb]{0.31,0.60,0.02}{#1}}
\newcommand{\VariableTok}[1]{\textcolor[rgb]{0.00,0.00,0.00}{#1}}
\newcommand{\VerbatimStringTok}[1]{\textcolor[rgb]{0.31,0.60,0.02}{#1}}
\newcommand{\WarningTok}[1]{\textcolor[rgb]{0.56,0.35,0.01}{\textbf{\textit{#1}}}}
\usepackage{graphicx,grffile}
\makeatletter
\def\maxwidth{\ifdim\Gin@nat@width>\linewidth\linewidth\else\Gin@nat@width\fi}
\def\maxheight{\ifdim\Gin@nat@height>\textheight\textheight\else\Gin@nat@height\fi}
\makeatother
% Scale images if necessary, so that they will not overflow the page
% margins by default, and it is still possible to overwrite the defaults
% using explicit options in \includegraphics[width, height, ...]{}
\setkeys{Gin}{width=\maxwidth,height=\maxheight,keepaspectratio}
\setlength{\emergencystretch}{3em}  % prevent overfull lines
\providecommand{\tightlist}{%
  \setlength{\itemsep}{0pt}\setlength{\parskip}{0pt}}
\setcounter{secnumdepth}{0}
% Redefines (sub)paragraphs to behave more like sections
\ifx\paragraph\undefined\else
\let\oldparagraph\paragraph
\renewcommand{\paragraph}[1]{\oldparagraph{#1}\mbox{}}
\fi
\ifx\subparagraph\undefined\else
\let\oldsubparagraph\subparagraph
\renewcommand{\subparagraph}[1]{\oldsubparagraph{#1}\mbox{}}
\fi

% set default figure placement to htbp
\makeatletter
\def\fps@figure{htbp}
\makeatother


\title{Class 6: R functions}
\author{Jasmin Revanna}
\date{1/24/2020}

\begin{document}
\maketitle

\hypertarget{this-is-level-2-heading}{%
\subsection{This is level 2 heading}\label{this-is-level-2-heading}}

This is \textbf{regular} old \emph{text}!

and a list of stuff

\begin{itemize}
\tightlist
\item
  toy
\item
  garage
\item
  shoe
\end{itemize}

\begin{Shaded}
\begin{Highlighting}[]
\KeywordTok{plot}\NormalTok{(}\DecValTok{1}\OperatorTok{:}\DecValTok{5}\NormalTok{, }\DataTypeTok{col=}\StringTok{"light blue"}\NormalTok{, }\DataTypeTok{type=}\StringTok{"o"}\NormalTok{)}
\end{Highlighting}
\end{Shaded}

\includegraphics{Class06_files/figure-latex/unnamed-chunk-1-1.pdf}

let's insert a code chunk with the shortcut \texttt{Option-CMD-i}:

\begin{Shaded}
\begin{Highlighting}[]
\NormalTok{x <-}\StringTok{ }\KeywordTok{c}\NormalTok{(}\DecValTok{1}\OperatorTok{:}\DecValTok{10}\NormalTok{)}
\NormalTok{x}
\end{Highlighting}
\end{Shaded}

\begin{verbatim}
##  [1]  1  2  3  4  5  6  7  8  9 10
\end{verbatim}

Here is my analysis of your new data. It looks ok.. the mean of your
data is 5.5.

We will use the read.table function

\begin{Shaded}
\begin{Highlighting}[]
\NormalTok{x <-}\StringTok{ }\KeywordTok{read.csv}\NormalTok{(}\StringTok{"test1.txt"}\NormalTok{)}
\end{Highlighting}
\end{Shaded}

\begin{Shaded}
\begin{Highlighting}[]
\NormalTok{y <-}\StringTok{ }\KeywordTok{read.table}\NormalTok{(}\StringTok{"test2.txt"}\NormalTok{, }\DataTypeTok{header =} \OtherTok{TRUE}\NormalTok{, }\DataTypeTok{sep =} \StringTok{"$"}\NormalTok{)}
\end{Highlighting}
\end{Shaded}

\begin{Shaded}
\begin{Highlighting}[]
\NormalTok{z <-}\StringTok{ }\KeywordTok{read.table}\NormalTok{(}\StringTok{"test3.txt"}\NormalTok{)}
\end{Highlighting}
\end{Shaded}

\hypertarget{our-first-function}{%
\subsection{Our first function}\label{our-first-function}}

This is an example function \texttt{add} withinputs a\texttt{x} and
\texttt{y}

\begin{Shaded}
\begin{Highlighting}[]
\NormalTok{add <-}\StringTok{ }\ControlFlowTok{function}\NormalTok{(x, }\DataTypeTok{y=}\DecValTok{1}\NormalTok{) \{}
 \CommentTok{# Sum the input x and y}
\NormalTok{ x }\OperatorTok{+}\StringTok{ }\NormalTok{y}
\NormalTok{\}}
\end{Highlighting}
\end{Shaded}

Let's try to use it

\begin{Shaded}
\begin{Highlighting}[]
\KeywordTok{add}\NormalTok{(}\DataTypeTok{x=}\KeywordTok{c}\NormalTok{(}\DecValTok{1}\NormalTok{, }\DecValTok{4}\NormalTok{, }\DecValTok{7}\NormalTok{), }\DataTypeTok{y=}\DecValTok{4}\NormalTok{)}
\end{Highlighting}
\end{Shaded}

\begin{verbatim}
## [1]  5  8 11
\end{verbatim}

\begin{Shaded}
\begin{Highlighting}[]
\NormalTok{rescale2 <-}\StringTok{ }\ControlFlowTok{function}\NormalTok{(x) \{}
\NormalTok{ rng <-}\KeywordTok{range}\NormalTok{(x, }\DataTypeTok{na.rm =} \OtherTok{TRUE}\NormalTok{)}
\NormalTok{ (x }\OperatorTok{-}\StringTok{ }\NormalTok{rng[}\DecValTok{1}\NormalTok{]) }\OperatorTok{/}\StringTok{ }\NormalTok{(rng[}\DecValTok{2}\NormalTok{] }\OperatorTok{-}\StringTok{ }\NormalTok{rng[}\DecValTok{1}\NormalTok{])}
\NormalTok{\}}
\end{Highlighting}
\end{Shaded}

\begin{Shaded}
\begin{Highlighting}[]
\KeywordTok{rescale2}\NormalTok{(}\KeywordTok{c}\NormalTok{(}\DecValTok{1}\OperatorTok{:}\DecValTok{10}\NormalTok{, }\OtherTok{NA}\NormalTok{))}
\end{Highlighting}
\end{Shaded}

\begin{verbatim}
##  [1] 0.0000000 0.1111111 0.2222222 0.3333333 0.4444444 0.5555556 0.6666667
##  [8] 0.7777778 0.8888889 1.0000000        NA
\end{verbatim}

~~~

\begin{Shaded}
\begin{Highlighting}[]
\NormalTok{rescale <-}\StringTok{ }\ControlFlowTok{function}\NormalTok{(x, }\DataTypeTok{na.rm=}\OtherTok{TRUE}\NormalTok{, }\DataTypeTok{plot=}\OtherTok{FALSE}\NormalTok{) \{}
 \ControlFlowTok{if}\NormalTok{(na.rm) \{}
\NormalTok{ rng <-}\KeywordTok{range}\NormalTok{(x, }\DataTypeTok{na.rm=}\OtherTok{TRUE}\NormalTok{)}
\NormalTok{ \} }\ControlFlowTok{else}\NormalTok{ \{}
\NormalTok{ rng <-}\KeywordTok{range}\NormalTok{(x)}
\NormalTok{ \}}
 \KeywordTok{print}\NormalTok{(}\StringTok{"Hello"}\NormalTok{)}
\NormalTok{ answer <-}\StringTok{ }\NormalTok{(x }\OperatorTok{-}\StringTok{ }\NormalTok{rng[}\DecValTok{1}\NormalTok{]) }\OperatorTok{/}\StringTok{ }\NormalTok{(rng[}\DecValTok{2}\NormalTok{] }\OperatorTok{-}\StringTok{ }\NormalTok{rng[}\DecValTok{1}\NormalTok{])}
 \KeywordTok{return}\NormalTok{(answer)}
 \KeywordTok{print}\NormalTok{(}\StringTok{"is it me you are looking for?"}\NormalTok{)}
 \ControlFlowTok{if}\NormalTok{(plot) \{}
 \KeywordTok{plot}\NormalTok{(answer, }\DataTypeTok{typ=}\StringTok{"b"}\NormalTok{, }\DataTypeTok{lwd=}\DecValTok{4}\NormalTok{)}
\NormalTok{ \}}
 \KeywordTok{print}\NormalTok{(}\StringTok{"I can see it in ..."}\NormalTok{)}
\NormalTok{\}}
\end{Highlighting}
\end{Shaded}

\begin{Shaded}
\begin{Highlighting}[]
\CommentTok{##rescale3(x)}
\end{Highlighting}
\end{Shaded}

\end{document}
